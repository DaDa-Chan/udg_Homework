\documentclass[fontset=fandol]{ctexart}  
\usepackage{geometry}  
\geometry{margin=1.5cm, vmargin={0pt,1cm}}  
\setlength{\topmargin}{-1cm}  
\setlength{\paperheight}{29.7cm}  
\setlength{\textheight}{25.3cm}  
  
\usepackage{xcolor}  % 使用 xcolor 而非 color  
\usepackage{listings}  
  
% 设置代码样式  
\lstset{  
    basicstyle=\ttfamily\small,  
    keywordstyle=\color{black},  
    commentstyle=\color{gray},  
    stringstyle=\color{red},  
    numbers=left,  
    numberstyle=\tiny\color{gray},  
    stepnumber=1,  
    numbersep=10pt,  
    backgroundcolor=\color{white},  
    showspaces=false,  
    showstringspaces=false,  
    frame=single,  
    rulecolor=\color{black},  
    tabsize=4,  
    captionpos=b,  
    breaklines=true,  
    breakatwhitespace=false,  
    escapeinside={(*@}{@*)}  % 更改逃逸字符设置  
}  
  
\title{List 类的改进与测试报告}  
\author{陈汉铮}  
\date{\today}  
  
\begin{document}  
  
\maketitle  
  
\section{List.h 结构的完善}  
\subsection{增加 const\_iterator 和 iterator 的 operator--} 
根据文件"List.h"我们发现pop\_back()实现依赖于递减运算符"--",pop\_back()的具体代码如下:

\begin{lstlisting}[language=C++, caption={pop\_back}]  
/** 
     * @brief 删除 List 的最后一个数据节点.
     * 
     */
    void pop_back()
    {
        erase(--end());
    } 
\end{lstlisting}


为了确保 pop\_back() 函数的实现,我们在 const\_iterator 和 iterator 中添加了递减运算符"--"。这允许迭代器向列表的前端移动,是 pop\_back() 函数能够正确执行的关键。以下是operator --的具体代码:  
  
\begin{lstlisting}[language=C++, caption={const\_iterator的前置自减运算符实现}]  
/**  
 * @brief 迭代器的前置自减运算符。用于将迭代器指向前一节点  
 *  
 * @return const_iterator 上一个节点的迭代器  
 */  
const_iterator &operator--()  
{  
    current = current->prev;  
    return *this;  
}  
\end{lstlisting}


\begin{lstlisting}[language=C++, caption={iterator的前置自减运算符实现}]  
/**
 * @brief 迭代器的前置自减运算符. 用于将迭代器指向上一个节点.
 *
 * @return iterator& 上一个节点的迭代器.
 */
iterator &operator--()
{
    this->current = this->current->prev;
    return *this;
}  
\end{lstlisting}

以及相应的后置自减运算符,具体代码如下:

\begin{lstlisting}[language=C++, caption={const\_iterator的后置自减运算符实现}]  
/**
 * @brief 迭代器的后置自减运算符。用于将迭代器指向前一节点
 * 
 * @return const_iterator 上一个节点的迭代器
 */
const_iterator operator--(int)
{
    const_iterator old = *this;
    --(*this);
    return old;
}  
\end{lstlisting}

\begin{lstlisting}[language=C++, caption={iterator的后置自减运算符实现}]  
/**
 * @brief 迭代器的后置自减运算符。用于将迭代器指向前一节点
 * 
 * @return iterator 上一个节点的迭代器
 */
iterator operator--(int)
{
    iterator old = *this;
    ++(*this);
    return old;
} 
\end{lstlisting}
至此,pop\_back函数中第七行代码"erase(--end());"得以实现,相应的功能也得以实现
  
\subsection{完善 erase 和 clear 函数}  
我们对 erase(iterator from, iterator to) 函数进行了改进,使其能够更高效地处理元素范围的删除。接下来,我们简写 erase(iterator from, iterator to) 为 erase(from,to),erase(iterator itr) 为 erase(itr)。

改善过后代码如下(包括原本代码):

\begin{lstlisting}[language=C++, caption={erase(iterator from, iterator to)函数}]  
iterator erase(iterator from, iterator to)
{
    /// 这里其实就是重复调用了单个节点删除的 erase 函数.
    /// 因此这个内循环和外部循环完全一致. 
    /// 这里可以做适当的优化么?
    //for( iterator itr = from; itr != to; )
    //    itr = erase( itr );
    //return to;
    if (from == to) return to;  // 如果范围为空,直接返回    
    Node* start = from.current;    
    
    Node* end = to.current;    
    
    if (!start || start == tail) return iterator(end);  // 额外的检查,确保不会操作尾部之后的元素    
    
    start->prev->next = end;    
    if (end) // 确保 end 不是 nullptr
    {      
        end->prev = start->prev;    
    }    
    
    while (start != end) 
    {    
        Node* next = start->next;    
        delete start;    
        start = next;    
        theSize--;    
    }    
    return iterator(end);  
}  
\end{lstlisting}
课堂上给出的erase(iterator from, iterator to) 函数是基于erase(iterator itr)这个函数实现的,这也是最直接的思想。然而我的改进的思路为:首先,利用链表的性质,直接将from所指节点和to所指的节点连接。再将中间的迭代器一一删除。

观察代码可以发现,虽然每一次 erase(itr) 不需要单独处理边界条件,但如果要删除多个连续节点,单独调用 erase(itr) 将导致多次不必要的链接更新。例如,如果删除一个范围内的所有节点,使用 erase(from, to) 只需要在操作的开始和结束时更新一次链接,而使用多次 erase(itr) 则每删除一个节点都需要更新链接,这在处理大量数据时效率较低。

在完善erase(from,to)这个函数之后,笔者发现即使实现了该函数的优化,但该函数却并没有实际的用处在原本的类内。个人认为这是价值的剩余。于是,笔者首先希望利用该函数对内部其他函数进行优化,遂发现了 clear() 这个函数,一下是原本的代码:

\begin{lstlisting}[language=C++, caption={clear() origin version}]  
void clear() {  
    while (!empty()) {  
        pop_front();  
    }  
} 
\end{lstlisting}
其中 pop\_front() 函数的实现是基于 erase(itr) 的实现。然而根据 clear() 函数的意义,显然是希望删除list中除了首尾节点以外的节点,本质上是一个范围。于是,笔者将 clear() 函数进行了改善,提高了代码实现的效率和简洁性,代码如下:

\begin{lstlisting}[language=C++, caption={clear() new version}]  
/**
 * @brief 清空 List 中的数据.
 * 
 */
void clear()
{
    if(!empty()) erase(begin(),end());//erase()函数内已经判断list是不是空的情况了
    //用erase(from,to)?
}
\end{lstlisting}

至此,笔者对原本的List.h的完善和优化已经结束,接下来介绍笔者增加的操作和功能。

\section{List.h 中链表操作的增加}  
为 List 类增加了三个新的功能,分别为printlist(), reverse() 和 remove\_all() 函数,这些功能增强了 List 类处理数据的能力。  
  
\subsection{printlist 函数}  
printlist() 函数提供了一种方便的方式来显示列表中的所有元素,这在开发和测试期间特别有用,可以用来调试和验证列表内容。也为我们后续设计测试代码提供了可视化工具和追踪工具。具体代码如下:

\begin{lstlisting}[language=C++, caption={printlist()函数}]  
/**
 * @brief 打印链表
 * @return 输出链表所有元素
 */
void printlist()
{
    if(empty() || head == nullptr) 
    {
        cout <<"The list is empty!"<< endl;
        return;
    }
    Node* p = head ->next;
    while(p != tail)
    {
        cout<< p -> data <<"\t";
        p =p->next;
    }
    cout << endl;
    return;
}

\end{lstlisting}
在这里,list如果是空,则输出"The list is empty!",否则则一一输出list内所有的值。
  
\subsection{reverse 函数}  
reverse() 函数允许就地反转列表,对于需要高效反转元素顺序的应用程序来说至关重要。代码如下:

\begin{lstlisting}[language=C++, caption={reverse()函数}]  
/**
 * @brief 反转列表
 * @return 反转后的链表
 */
void reverse()
{
    if(empty() || theSize == 1)//如果链表没有元素或者只有一个元素,那么链表反转后就是它本身
    {
        return;
    }
    Node* p = head -> next;
    Node*tmp = p;
    while(tmp != tail)//step1 把除head和tail以外的节点交换next,prev
    {
        p = p->next;
        tmp ->next = tmp -> prev;
        tmp ->prev = p;
        tmp = p;
    }
    tmp = head ->next;//重新连接head和tail
    head ->next ->next =tail;
    tail ->prev ->prev =head;
         
    head ->next = tail ->prev;
    tail ->prev = tmp;  
    tmp = nullptr;     

}
\end{lstlisting}

对于这个函数的实现,笔者的思路是这样的:首先将除了head和tail两个哨兵节点以外的所有节点,进行next和prev指针交换的操作。然后,再将原本第一个节点的next和原本的最后一个节点的prev与tail和head重连。最后再修改head和tail与链表的连接。

  
\subsection{remove\_all 函数}  
remove\_all 函数通过提供一种能够删除所有特定元素的出现的方法,增强了 List 的功能,这对于管理可能需要清理重复值的数据集非常重要。代码如下:

\begin{lstlisting}[language=C++, caption={remove\_all()函数}]  
void remove_all(const Object &val)
{
    iterator it = begin();
    iterator start;

    while(it != end())
    {
        while(it != end() && *it != val)//找到第一个data = val 的节点的迭代器
        {
            ++it;
        }
        if(it == end()) break;//如果没有找到则退出

        start = it;
        while(it!=end() && *it == val)
        {
            ++it;
        }
        erase(start,it);
    }
}
\end{lstlisting}

该函数实现的思路,也是基于 erase(from,to) 和clear()一样,该函数同样可以基于  erase(itr) 函数来实现,但是考虑到要删除的值可能连续出现,利用erase(from,to) 函数显然效率更高。

\section{设计和测试 List.cpp}  
\subsection{List.cpp 的设计和结果}  
在设计 List.cpp 时,我们的目标是通过一系列测试用例全面验证 List 类的所有功能。这包括基本的添加、删除操作,以及新实现的功能如列表反转和元素清除等。笔者输入的指令为"g++ -fsanitize=address -fno-omit-frame-pointer -g -std=c++17 List.cpp -o List"由于系统原因无法使用Valgrind,但AddressSanitizer同样能够检查是否存在内存泄漏。如果不存在,则没有任何显示。

\subsubsection{测试构造函数和基本操作}
首先我们先测试 list 的构造函数,和一些基本的操作,包括但不限于:front(),back(),size(),printlist().代码如下:
\begin{lstlisting}[language=C++, caption={测试构造函数和基本操作}]  
// 测试构造函数和基本操作  
    cout << "Testing default constructor and push_back:" << endl;  
    List<int> list1;  
    for (int i = 1; i <= 5; ++i) {  
        list1.push_back(i);  
    }  
    list1.printlist();  // 输出应为 1 2 3 4 5  
    cout<<"list1's size is:"<<list1.size()<<endl;
    cout<<"list1's the first value is:"<<list1.front()<<endl;
    cout<<"list1's the last value is:"<<list1.back()<<endl;

\end{lstlisting}

测试输出的结果为:
\begin{verbatim}  
Testing default constructor and push_back:  
1       2       3       4       5  
list1's size is: 5  
list1's the first value is: 1  
list1's the last value is: 5  
\end{verbatim}  

\subsubsection{测试初始化列表构造函数}
测试代码如下:
\begin{lstlisting}[language=C++, caption={测试初始化列表构造函数}]  
// 测试初始化列表构造函数  
    cout << "Testing initializer list constructor:" << endl;  
    List<int> list2 {10, 20, 30, 40, 50};  
    list2.printlist();  // 输出应为 10 20 30 40 50
\end{lstlisting}

测试结果如下:
\begin{verbatim}  
Testing initializer list constructor:
10      20      30      40      50 
\end{verbatim}  

\subsubsection{测试拷贝构造函数}
测试代码如下:
\begin{lstlisting}[language=C++, caption={测试拷贝构造函数}]  
// 测试拷贝构造函数
    cout << "Testing copy constructor:" << endl;  
    List<int> list3 = list2;  
    list3.printlist();  // 输出应为 10 20 30 40 50 
\end{lstlisting}

测试结果如下:
\begin{verbatim}  
Testing copy constructor:
10      20      30      40      50
\end{verbatim} 

\subsubsection{测试移动构造函数}
本节测试移动构造函数,并顺带测试了 empty() 。同时移动构造后,对list3进行printlist()操作,其实也侧面验证了list3被析构后的结果。测试代码如下:
\begin{lstlisting}[language=C++, caption={测试移动构造函数}]  
// 测试构造函数和基本操作  
    cout << "Testing move constructor:" << endl;  
    List<int> list4 = std::move(list3);  
    list4.printlist();  // 输出应为 10 20 30 40 50  
    cout << "After move, list3 is " << (list3.empty() ? "empty" : "not empty") << endl;  // 应为空
\end{lstlisting}

测试结果如下:
\begin{verbatim}  
Testing move constructor:
10      20      30      40      50
After move, list3 is empty
\end{verbatim}  

\subsubsection{测试移动赋值运算符}
测试代码如下:
\begin{lstlisting}[language=C++, caption={测试移动赋值运算符}]  
// 测试移动赋值运算符
    cout << "Testing move assignment operator:" << endl;  
    List<int> list5;  
    list5 = std::move(list4);  
    list5.printlist();  // 输出应为 10 20 30 40 50  
    cout << "After move assignment, list4 is " << (list4.empty() ? "empty" : "not empty") << endl;  // 应为空 
\end{lstlisting}

测试结果如下:
\begin{verbatim}  
Testing move assignment operator:
10      20      30      40      50
After move assignment, list4 is empty
\end{verbatim}

\subsubsection{测试赋值运算符}
测试代码如下:
\begin{lstlisting}[language=C++, caption={测试赋值运算符}]  
// 测试赋值运算符  
    cout << "Testing assignment operator:" << endl;  
    list1 = list2;  
    list1.printlist();  // 输出应为 10 20 30 40 50
\end{lstlisting}

测试结果如下:
\begin{verbatim}  
Testing move assignment operator:
10      20      30      40      50
\end{verbatim}

\subsubsection{测试push\_front, pop\_back, pop\_front}
测试代码如下:
\begin{lstlisting}[language=C++, caption={测试赋值运算符}]  
// 测试push_front, pop_back, pop_front  
    cout << "Testing push_front, pop_back, and pop_front:" << endl;  
    list1.push_front(0);  
    list1.pop_back();  
    list1.pop_front();  
    list1.printlist();  // 输出应为 10 20 30 40
\end{lstlisting}

测试结果如下:
\begin{verbatim}  
Testing push_front, pop_back, and pop_front:
10      20      30      40
\end{verbatim}

\subsubsection{测试插入和删除}
测试代码如下:
\begin{lstlisting}[language=C++, caption={测试插入和删除}]  
// 测试插入和删除  
    cout << "Testing insert and erase:" << endl;  
    auto it = list1.begin();  
    ++it;  // 指向 20  
    list1.insert(it, 15);  // 在 20 前插入 15  
    list1.printlist();  // 输出应为 10 15 20 30  
    list1.erase(it);  // 删除 20  
    list1.printlist();  // 输出应为 10 15 30
\end{lstlisting}

测试结果如下:
\begin{verbatim}  
Testing insert and erase:
10      15      20      30      40
10      15      30      40
\end{verbatim}

\subsubsection{测试范围删除}
测试代码如下:
\begin{lstlisting}[language=C++, caption={测试范围删除}]  
// 测试范围删除  
    cout << "Testing range erase:" << endl;  
    list1.insert(list1.end(), 20);  
    list1.insert(list1.end(), 30);  
    list1.printlist();  // 输出应为 10 15 30 20 30  
    it = list1.begin();  
    ++it;  // 指向 15  
    auto it2 = list1.end();    
    list1.erase(it, it2);  
    list1.printlist();  // 输出应为 10
\end{lstlisting}

测试结果如下:
\begin{verbatim}  
Testing range erase:
10      15      30      40      20      30
10
\end{verbatim}

\subsubsection{测试反转}
测试代码如下:
\begin{lstlisting}[language=C++, caption={测试反转}]  
// 测试反转  
    cout << "Testing reverse:" << endl;  
    list1.push_back(10);  
    list1.push_back(20);  
    list1.push_back(30);  
    list1.printlist();  // 输出应为 10 10 20 30  
    list1.reverse();  
    list1.printlist();  // 输出应为 30 20 10 10
\end{lstlisting}

测试结果如下:
\begin{verbatim}  
Testing reverse:
10      10      20      30
30      20      10      10
\end{verbatim}

\subsubsection{测试remove\_all}
测试代码如下:
\begin{lstlisting}[language=C++, caption={测试remove\_all}]  
// 测试remove_all  
    cout << "Testing remove_all:" << endl;  
    list1.remove_all(10);  
    list1.printlist();  // 输出应为 30 20
\end{lstlisting}

测试结果如下:
\begin{verbatim}  
Testing remove_all:
30      20
\end{verbatim}

到这里
\end{document} 

