\documentclass[fontset=fandol]{ctexart}  

\usepackage{geometry}  
\geometry{margin=1.5cm, vmargin={0pt,1cm}}  
\setlength{\topmargin}{-1cm}  
\setlength{\paperheight}{29.7cm}  
\setlength{\textheight}{25.3cm}  
  
\usepackage{xcolor}  % 使用 xcolor 而非 color  
\usepackage{listings}  
  
% 设置代码样式  
\lstset{  
    basicstyle=\ttfamily\small,  
    keywordstyle=\color{black},  
    commentstyle=\color{gray},  
    stringstyle=\color{red},  
    numbers=left,  
    numberstyle=\tiny\color{gray},  
    stepnumber=1,  
    numbersep=10pt,  
    backgroundcolor=\color{white},  
    showspaces=false,  
    showstringspaces=false,  
    frame=single,  
    rulecolor=\color{black},  
    tabsize=4,  
    captionpos=b,  
    breaklines=true,  
    breakatwhitespace=false,  
    escapeinside={(*@}{@*)}  % 更改逃逸字符设置  
}  
% 在hyperref包选项中添加unicode=true
\usepackage[colorlinks=true, linkcolor=blue, citecolor=blue, urlcolor=blue, 
            bookmarks=true, unicode=true]{hyperref}
% 在hyperref之后加载bookmark包
\usepackage{bookmark}

\title{BST\_remove的改善}
\author{汉铮 陈}
\date{\today}

\begin{document}

\maketitle

\section{remove函数的完善}
这与部分主要分为三个主要内容:辅助函数 detachMin( BinaryNode* \&t)的构造、
remove(const Comparable \&x, BinaryNode * \&t)函数的完善、树的简单可视化打印drawTree(BinaryNode* t ,int space , bool isLeft)的实现。
以下将上述函分别简写为 detachMin() 、remove() 、 drawTree()

\subsection{detchMin()函数的构造}
在这里构造辅助函数detachMin()的主要作用就是查找以 t 为根的子树中的最小节点,返回这个节点,并从原子树中删除这个节点。显然,当要删除的节点具有两个子树时,通过这个函数返回的右子树最小节点将代替被删除节点。\par
该函数实现的基本原理与findMin()函数类似,只需要需要利用递归的方法找到右子树的最小节点,并构造一个临时节点记住该节点的值以及指针,最后把最小节点删掉即可。具体代码如下:

\begin{lstlisting}[language=C++, caption={detachMin()函数}]  
/**
 * @brief 查找以 t 为根的子树中的最小节点,返回这个节点,并从原子树中删除这个节点
 * @param t 当前节点指针
 * @return 以t 为根的子树中的最小节点
 */
BinaryNode* detachMin(BinaryNode *&t)
{
    if(!t) return nullptr;

    if(!t ->left)//判断是否为叶子节点
    {
        BinaryNode* minNode = t;
        t = t->right;
        return minNode; 
    }
    else
    {
        return detachMin(t -> left);
    }
}
\end{lstlisting}

而findMin()函数的代码如下:
\begin{lstlisting}[language=C++, caption={findMin()函数}]  
BinaryNode *findMin(BinaryNode *t) const {
    /// 从一个空节点开始查找,返回空指针                
    if (t == nullptr) 
    {
        return nullptr;
    }
    /// 向左无路了,当前节点就是最小元素
    if (t->left == nullptr) 
    {
        return t;
    }
    /// 否则继续向左查找
    return findMin(t->left);
}
\end{lstlisting}

\subsection{remove()函数的完善}
要通过移动节点的方式实现remove操作总体分为四步:1.找到要删除的节点 cur 和它的父节点 par; 2.找到 cur的右子树中的最小值节点 min ,并将其断开 ; 3.将 min的左子树和右子树分别链接到 cur 的左子树和右子树;4.将 par的指针指向 min,然后删除 cur。具体代码如下:
\begin{lstlisting}[language=C++, caption={remove()函数}]  
void remove(const Comparable &x, BinaryNode * &t) 
{
    if(!t) return;//如果t为空节点

    BinaryNode* par = nullptr;
    BinaryNode* cur = t;
    while(cur && cur -> element != x )//找到值为x的节点以及其parent节点,注意条件顺序
    {
        par =cur;
        if(cur -> element < x )
        {
            cur = cur->right;
        }else if(cur -> element > x)
        {
            cur = cur->left;
        }
    }

    if(!cur) return;//x没有被找到

    if(!(cur ->left) || !(cur -> right))//要删除的节点至多只有一个child节点
    {
        if(!par)//如果删除根节点
        {
            t = (cur -> left)? cur->left : cur->right;//修改根节点
        }
        if(par -> left == cur)//t是par的左节点
        {
            par -> left = (cur -> left)? cur->left : cur->right;
        }else
        {
            par -> right = (cur -> left)? cur->left : cur->right;
        }
        delete cur;
    }else//有两个child节点
    {
        BinaryNode* min = detachMin(cur->right);

        min -> left = cur -> left;// 将当前节点的左子树链接到后继节点
        min -> right = cur -> right;// 将当前节点的右子树链接到后继节点
        if(!par)//如果删除根节点
        {
            t = min;//修改根节点
        }
        else if(par -> left == t)//t是par的左节点
        {
            par -> left = min;// 父节点指向后继节点
        }else//t是右节点
        {
            par -> right = min;// 父节点指向后继节点
        }
        delete cur;
    }
}
\end{lstlisting}


\subsection{树的可视化实现}
在这里构造drawTree()函数主要是为了,便于直观的观察和跟踪remove()函数操作后的结果,这里打印的树的节点是纵向排列的。具体代码如下:
\begin{lstlisting}[language=C++, caption={draw()函数}]  
void drawTree(BinaryNode* node, int space , bool isLeft) {
    if (!node) return;

    space += 10; // 增加空间以便于显示

    // 先打印右子树
    drawTree(node->right, space , false);

    // 打印当前节点
    std::cout << std::endl;
    for (int i = 10; i < space; i++) {
        std::cout << ' '; // 输出空格
    }

    if (isLeft) {
        std::cout << " \\"; // 左子树的枝
    } else if (space > 10) {
        std::cout << " / "; // 右子树的枝
    }

    std::cout << " " << node->element << "\n"; // 打印节点值

    // 打印左子树
    drawTree(node->left, space , true);
}
\end{lstlisting}





\section{测试程序的设计}
这部分主要设计的是test.cpp部分,验证remove()函数的正确性。\par
首先构造一个BST:
\begin{lstlisting}[language=C++, caption={testBinarySearchTree()测试函数}]  
#include <iostream>
#include "BST.h"  // 假设 BinarySearchTree 类定义在这个头文件中
using namespace std;

int main() {
    BinarySearchTree<int> bst;

    // 测试插入功能
    bst.insert(50);
    bst.insert(30);
    bst.insert(70);
    bst.insert(20);
    bst.insert(40);
    bst.insert(60);
    bst.insert(80);

    std::cout << "before any deletions: ";
    bst.drawTree();

    // 删除叶子节点
    cout << "\nDeleting leaf node (20): ";
    bst.remove(20);
    bst.drawTree();

    // 删除只有一个子节点的节点
    cout << "\nDeleting node with one child (30): ";
    bst.remove(30);
    bst.drawTree();

    // 删除具有两个子节点的节点
    cout << "\nDeleting node with two children (50): ";
    bst.remove(50);
    bst.drawTree();

    cout<<endl;
    // 测试清空树
    bst.makeEmpty();
    cout << "Tree after making empty:" << std::endl;
    bst.printTree();

    return 0;
}
\end{lstlisting}
构造树的输出结果如下:
\begin{verbatim}  
before any deletions: 
                     /  80

           /  70

                     \ 60

 50

                     /  40

           \ 30

                     \ 20

\end{verbatim}

删除叶子节点后的树形状为:
\begin{verbatim}  
Deleting leaf node (20): 
                     /  80

           /  70

                     \ 60

 50

                     /  40

           \ 30
\end{verbatim}
删除只有一个子节点的节点后,树的形状为:
\begin{verbatim}  
Deleting node with one child (30): 
                     /  80

           /  70

                     \ 60

 50

           \ 40
\end{verbatim}
删除具有两个子节点的节点(这里是root)后,树的形状为:
\begin{verbatim}  
Deleting node with two children (50): 
                     /  80

           /  70

 60

           \ 40
\end{verbatim}
最后清空树:
\begin{verbatim}  
Tree after making empty:
Empty tree
\end{verbatim}

至此测试程序设计结束,笔者输入的指令为"g++ -fsanitize=address -fno-omit-frame-pointer -g -std=c++17 test\_BST.cpp -o test"利用AddressSanitizer检查是否存在内存泄漏。如果不存在,则没有任何显示。

欢迎批评指正

\end{document}


